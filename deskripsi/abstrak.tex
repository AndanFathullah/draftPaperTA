% Mengubah keterangan `Abstract` ke bahasa indonesia.
% Hapus bagian ini untuk mengembalikan ke format awal.
\renewcommand\abstractname{Abstrak}

\begin{abstract}

  % Ubah paragraf berikut sesuai dengan abstrak dari penelitian.
  Fitur wajah seperti identifikasi umur, gender dan etnik dapat sangat berguna 
  dalam banyak pengimplementasian ilmu seperti pengamatan visual, diagnosa medis, 
  sistem interaksi komputer manusia, biometric, pengumpulan informasi, penegakan 
  hukum, pemasaran dan banyak lainnya. Dimana sebagian besar data mengenai fitur 
  wajah tersebut masih diambil secara manual melalui survei ataupun pengamatan 
  pada banyak individu. Berdasarkan World Population Clock pada websitenya, di 
  dunia terdapat lebih dari 7 miliar orang yang tersebar di berbagai macam pulau 
  dan benua. Jumlah tersebut masih terus bertambah sampai sekarang. Dimana di 
  setiap benua dan negara tersebut terdapat berbagai karakteristik dan ciri 
  manusia yang berbeda dengan kata lain Etnik yang berbeda-beda. Dengan banyaknya 
  jumlah penduduk dan keberagamannya tersebut, jika data fitur wajah diambil 
  secara manual akan memakan waktu dan tenaga yang banyak. Oleh karena itu perlu 
  dibuat suatu sistem yang dapat mengestimasi umur, gender dan etnik serta 
  menyimpan penghitungan datanya untuk mempermudah pengumpulan data. Dimana 
  kamera akan menangkap gambar dari seseorang dan dilakukan proses estimasi 
  umur, gender dan etnik yang kemudian datanya disimpan untuk digunakan kedepannya.

\end{abstract}

% Mengubah keterangan `Index terms` ke bahasa indonesia.
% Hapus bagian ini untuk mengembalikan ke format awal.
\renewcommand\IEEEkeywordsname{Kata kunci}

\begin{IEEEkeywords}

  % Ubah kata-kata berikut sesuai dengan kata kunci dari penelitian.
  \emph{CNN}, Umur, Gender, Ras, Citra, Wajah.

\end{IEEEkeywords}
