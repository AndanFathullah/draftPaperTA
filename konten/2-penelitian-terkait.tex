% Ubah judul dan label berikut sesuai dengan yang diinginkan.
\section{Penelitian Terkait}
\label{sec:penelitianterkait}

% Ubah paragraf-paragraf pada bagian ini sesuai dengan yang diinginkan.

Berberapa penelitian yang telah dilakukan terkait dengan judul Tugas Akhir ini antara lain dilakukan oleh 
A. Garain et al. \textit{GRANet A Deep Learning Model for Classification of Age and GenderFrom Facial Images}. 
Dimana pada penelitian tersebut mencoba menggunakan beberapa dataset seperti wikipedia age dataset, 
FG-Net, AFAD, AduenceDB dataset dan UTKFace dataset serta menggunakan model yang arsitekturnya seperti 
\textit{Residual Attention Network} dengan tambahan parameter \textit{Gate} seperti pada \textit{Gated Residual Units (GRU’s)}. 
Yang berhasil melakukan prediksi umur dan gender dengan baik. Namun belum menggunakan pendeteksian etnik\citep{Granet}. 
Penelitian lainnya dilakukan oleh G. Guo et al. dengan judul \textit{Human Age Estimation What is the Influence 
Across Race and Gender} yang menggunakan database MORPH-II dengan data gambar wajah sebanyak 55.000. 
Mereka membandingkan hasil estimasi umur antara individu dengan sesama etnik dan dengan yang berbeda 
etnik. Didapatkan tingkat eror yang signifikan pada percobaan estimasi individu  yang  berbeda etnik\citep{AgeGender}. 
Kemudian penelitian oleh M. Shin et al. \textit{Face Image-Based Age and Gender Estimation with Consideration of 
Ethnic Difference}, pada penelitian ini menggunakan \textit{CNN} dan \textit{SVM} yang berfungsi memisahkan dua etnik 
menjadi Asia dan Non Asia yang kemudian hasilnya digunakan untuk mendeteksi umur dan gender dari wajah 
yang diberikan. Dihasilkan bahwa pemisahan etnik dapat meningkatkan keakuratan estimasi umur, namun tidak 
berpengaruh pada gender\citep{HumanAgeEst}.
